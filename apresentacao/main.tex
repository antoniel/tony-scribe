\documentclass[aspectratio=169,xcolor=table]{beamer}
\usepackage{algorithm}
\usepackage{algpseudocode}
\usepackage[utf8]{inputenc}
\usepackage[T1]{fontenc}
\usepackage{lipsum, lmodern}
\usepackage{csquotes}
\usepackage{xcolor}
\usepackage[portuguese]{babel}

% ------------------------------------------------
% Tema e Configurações do Beamer
% ------------------------------------------------
\usetheme{DCC}

% Ajuste de espaçamento entre itens
\setbeamertemplate{itemize items}[circle]
\setbeamertemplate{itemize subitem}[circle]
\setlength{\itemsep}{0.8em}
\setlength{\parskip}{0.5em}

\graphicspath{{imgs/}{./imgs/}}

\author[Sousa, Antoniel; Sena, Luis Felipe; Farias, Claudio; Leahy, João]{%
  \textbf{Antoniel} \and \textbf{Luis Sena} \and \textbf{Claudio} \and \textbf{João Leahy}
}
\title{Anota.ai: Seu Caderno Inteligente de Aulas}
\subtitle{Sistema Web de Apoio ao Aprendizado - UFBA}
\institute{Universidade Federal da Bahia \\ Instituto de Computação}
\date{\today}

\begin{document}

%-------------------------------------------------
%  SLIDE DE TÍTULO
%-------------------------------------------------
\begin{frame}[plain,noframenumbering]
    \titlepage
\end{frame}

%-------------------------------------------------
%  SLIDE DE AGENDA
%-------------------------------------------------
\begin{frame}{Agenda}
    \tableofcontents
\end{frame}

\setlength{\parskip}{1em}

%=================================================
\section{Introdução}
%=================================================
\begin{frame}{O Problema}
    \vspace{0.5em}
    \begin{itemize}
        \item Durante aulas universitárias, é comum que alunos \textbf{percam informações importantes}
        \item Ritmo acelerado das explicações e distrações
        \item Materiais dispersos entre slides, Moodle e anotações pessoais
        \item \textbf{Dificuldade de revisão} e organização do conteúdo
    \end{itemize}
\end{frame}

\begin{frame}{A Solução: Anota.ai}
    \vspace{0.5em}
    \begin{center}
        \Large Uma plataforma inteligente de apoio ao aprendizado
    \end{center}
    
    \vspace{1em}
    
    \begin{itemize}
        \item Captura e organiza automaticamente o conteúdo das aulas
        \item Integra gravação de áudio, slides e anotações automáticas
        \item Utiliza Inteligência Artificial para gerar resumos e materiais de estudo
        \item Centraliza todo o material em um caderno digital pesquisável
    \end{itemize}
\end{frame}

%=================================================
\section{Equipe e Documentação}
%=================================================
\begin{frame}{Equipe e Responsabilidades}
    \begin{itemize}
        \item \textbf{Antoniel}
        \begin{itemize}
            \item Backend (Tanstack Start + Hono), Integração com APIs de IA, Banco de Dados
        \end{itemize}
        
        \item \textbf{Luis Sena}
        \begin{itemize}
            \item Frontend (React/Tanstack Start), Design de Interface, UX
        \end{itemize}
        
        \item \textbf{João Leahy}
        \begin{itemize}
            \item Infraestrutura, Deploy, Integração de Áudio/Transcrição
        \end{itemize}
        
        \item \textbf{Claudio}
        \begin{itemize}
            \item Testes, Documentação, Features de Colaboração
        \end{itemize}
    \end{itemize}
\end{frame}

\begin{frame}{Gestão de Projeto}
    \vspace{0.5em}
    \begin{itemize}
        \item \textbf{Ferramenta de Gestão:} TODO.md na raiz do repositório
        \begin{itemize}
            \item Acompanhamento de tarefas e sprints
            \item Registros de decisões técnicas
            \item Checklist de features e bugs
        \end{itemize}
        
        \vspace{1em}
        
        \item \textbf{Repositório:} GitHub (compartilhado com o professor)
        
        \vspace{1em}
        
        \item \textbf{Metodologia:} Desenvolvimento ágil com reuniões semanais
    \end{itemize}
\end{frame}

%=================================================
\section{Tecnologias e Framework}
%=================================================
\begin{frame}{Framework: React com Tanstack Start}
    \vspace{0.5em}
    \begin{center}
        \Large Por que React + Tanstack Start?
    \end{center}
    
    \vspace{1em}
    
    \begin{itemize}
        \item \textbf{Componentes Reutilizáveis:} Facilita a manutenção e escalabilidade
        \item \textbf{Ecossistema Rico:} Bibliotecas para roteamento, estado, UI, etc.
        \item \textbf{Performance:} Virtual DOM otimiza atualizações da interface
        \item \textbf{Tanstack Start:} Framework full-stack moderno com SSR e rotas dinâmicas
        \item \textbf{Type-safe:} Tipagem end-to-end do frontend ao backend
        \item \textbf{Comunidade Ativa:} Suporte amplo e documentação extensa
    \end{itemize}
\end{frame}

\begin{frame}{Stack Tecnológica Completa}
    \begin{columns}[T]
        \begin{column}{0.48\textwidth}
            \textbf{Frontend}
            \begin{itemize}
                \item Tanstack Start (React)
                \item TypeScript
                \item TailwindCSS
                \item Shadcn/ui
                \item TipTap Editor
            \end{itemize}
        \end{column}
        
        \begin{column}{0.48\textwidth}
            \textbf{Backend}
            \begin{itemize}
                \item Hono (Web Framework)
                \item PostgreSQL
                \item Drizzle ORM
                \item OAuth2/SSO UFBA
            \end{itemize}
        \end{column}
    \end{columns}
    
    \vspace{1.5em}
    
    \textbf{Serviços de IA}
    \begin{itemize}
        \item Whisper/WhisperX (Transcrição de áudio)
        \item Google Gemini (Resumos, flashcards e questões)
        \item OCR para extração de texto de slides
    \end{itemize}
\end{frame}

%=================================================
\section{Inovação e Funcionalidades}
%=================================================
\begin{frame}{Inovações do Anota.ai}
    \vspace{0.5em}
    \textbf{O que trazemos de novo?}
    
    \vspace{1em}
    
    \begin{itemize}
        \item \textbf{Diarização de Falas:} Separa automaticamente as vozes de professor e estudantes
        \item \textbf{Linha do Tempo Interativa:} Sincroniza áudio, slides e anotações em uma única linha
        \item \textbf{Resumos Automáticos:} Síntese textual e geração de flashcards com IA
        \item \textbf{Geração de Questões:} Cria perguntas objetivas e discursivas baseadas na transcrição
        \item \textbf{Busca Semântica:} Encontra conceitos por significado, não apenas palavras-chave
    \end{itemize}
\end{frame}

\begin{frame}{Funcionalidades Principais}
    \begin{columns}[T]
        \begin{column}{0.48\textwidth}
            \textbf{Captura}
            \begin{itemize}
                \item Gravação de áudio em tempo real
                \item Upload de slides/PDFs
                \item OCR para extração de texto
                \item Sincronização automática
            \end{itemize}
        \end{column}
        
        \begin{column}{0.48\textwidth}
            \textbf{Estudo}
            \begin{itemize}
                \item Resumos automáticos
                \item Flashcards gerados por IA
                \item Questões de revisão
                \item Busca inteligente
            \end{itemize}
        \end{column}
    \end{columns}
    
    \vspace{1.5em}
    
    \textbf{Colaboração}
    \begin{itemize}
        \item Compartilhamento com colegas e monitores
        \item Anotações colaborativas
        \item Caderno digital centralizado
    \end{itemize}
\end{frame}

%=================================================
\section{Requisitos do Sistema}
%=================================================
\begin{frame}{Requisitos Funcionais}
    \vspace{0.5em}
    \begin{itemize}
        \item \textbf{RF1:} Capturar áudio em tempo real durante as aulas
        \item \textbf{RF2:} Transcrever automaticamente usando Whisper/WhisperX
        \item \textbf{RF3:} Extrair textos de slides e imagens via OCR
        \item \textbf{RF4:} Associar transcrições e slides em linha do tempo única
        \item \textbf{RF5:} Permitir busca semântica em todo o conteúdo
        \item \textbf{RF6:} Gerar resumos automáticos e flashcards
        \item \textbf{RF7:} Permitir compartilhamento com monitores e colegas
    \end{itemize}
\end{frame}

\begin{frame}{Requisitos Não Funcionais e Regras de Negócio}
    \vspace{0.5em}
    \textbf{Requisitos Não Funcionais}
    \begin{itemize}
        \item \textbf{RNF1:} Funcionar em modo offline (conexão instável)
        \item \textbf{RNF2:} Garantir respostas em até 3 segundos
        \item \textbf{RNF3:} Segurança e privacidade dos dados de aula
        \item \textbf{RNF4:} Autenticação via SSO/UFBA
    \end{itemize}
    
    \vspace{1em}
    
    \textbf{Regras de Negócio}
    \begin{itemize}
        \item Apenas usuários autenticados (UFBA) podem acessar
        \item Monitores têm permissões de moderação
        \item Professores podem compartilhar materiais diretamente
    \end{itemize}
\end{frame}

%=================================================
\section{Entregas e Demonstração}
%=================================================
\begin{frame}{Requisitos de Entrega}
    \vspace{0.5em}
    \begin{enumerate}
        \item \textbf{Documentação Mínima do Sistema}
        \begin{itemize}
            \item TODO.md na raiz do repositório para gestão de tarefas
            \item Funções e responsabilidades de cada integrante
        \end{itemize}
        
        \vspace{0.5em}
        
        \item \textbf{Entrega do Protótipo do Sistema}
        \begin{itemize}
            \item Sistema funcional hospedado
            \item Código-fonte no GitHub
        \end{itemize}
        
        \vspace{0.5em}
        
        \item \textbf{Apresentação do Front-end}
        \begin{itemize}
            \item Interface desenvolvida em React/Tanstack Start
        \end{itemize}
        
        \vspace{0.5em}
        
        \item \textbf{Demonstração Prática do Framework}
        \begin{itemize}
            \item Exemplos de componentes React e hooks
        \end{itemize}
    \end{enumerate}
\end{frame}

\begin{frame}{Demonstração Prática: React com Tanstack Start}
    \vspace{0.5em}
    \textbf{Exemplo de Componentes e Features}
    
    \vspace{0.5em}
    
    \begin{itemize}
        \item \textbf{Editor de Anotações:} Componente TipTap integrado
        \item \textbf{Player de Áudio:} Controle de reprodução sincronizado
        \item \textbf{Linha do Tempo:} Visualização interativa de tópicos
        \item \textbf{Hooks Customizados:}
        \begin{itemize}
            \item \texttt{useNote()} - Gerenciamento de anotações
            \item \texttt{usePatient()} - Controle de pacientes/aulas
            \item \texttt{useStats()} - Estatísticas de uso
        \end{itemize}
        \item \textbf{API Routes:} Rotas type-safe com Hono
    \end{itemize}
    
    \vspace{0.5em}
    
    \textbf{Demonstração ao vivo durante a apresentação}
\end{frame}

%=================================================
\section{Conclusão}
%=================================================
\begin{frame}{Conclusão}
    \vspace{0.5em}
    
    O \textbf{Anota.ai} propõe uma solução completa e inovadora para:
    
    \vspace{1em}
    
    \begin{itemize}
        \item \textbf{Digitalização} e aprimoramento do aprendizado acadêmico
        \item \textbf{Unificação} de áudio, slides, anotações e IA
        \item \textbf{Redução} de perdas de conteúdo durante aulas
        \item \textbf{Melhoria} do estudo autônomo
        \item \textbf{Integração} entre alunos, monitores e professores
    \end{itemize}
    
    \vspace{1em}
    
    \begin{center}
        \Large Uma ferramenta para potencializar o aprendizado na UFBA
    \end{center}
\end{frame}

%=================================================
\section{Perguntas e Comentários}
%=================================================
\begin{frame}{Perguntas e Comentários}
    \begin{center}
        \Huge Obrigado!
    \end{center}
\end{frame}

\end{document}
